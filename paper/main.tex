\documentclass[english,submission]{programming}
\usepackage[backend=biber]{biblatex}
\addbibresource{refs.bib}
\usepackage{svg}
\svgpath{img}

% Fix broken footnotes
\makeatletter
\renewcommand\@makefntext[1]{\leftskip=0.7em\hskip-0.7em\@makefnmark\hspace{1mm}#1}
\makeatother
\renewcommand{\footnotesize}{\small}
\setlength{\footnotesep}{3mm}

\definecolor{darkblue}{rgb}{0,0,0.5}
\definecolor{darkgreen}{rgb}{0,0.3,0}
\definecolor{darkpink}{rgb}{0.4,0,0.3}
\definecolor{graygreen}{rgb}{0.3,0.5,0.3}
\definecolor{grayblue}{rgb}{0.2,0.2,0.6}
\definecolor{grayred}{rgb}{0.5,0.2,0.2}

\lstset{
  backgroundcolor=\color{white},     % choose the background color; you must add \usepackage{color} or \usepackage{xcolor}; should come as last argument
  % identifierstyle=\color{red},
  basicstyle=\normalsize\ttfamily,      % the size of the fonts that are used for the code
  breakatwhitespace=false,              % sets if automatic breaks should only happen at whitespace
  breaklines=false,                     % sets automatic line breaking
  captionpos=b,                         % sets the caption-position to bottom
  abovecaptionskip=-3 mm,
  commentstyle=\itshape\color{graygreen}, % comment style
  % escapeinside={(:}{:)},             % if you want to add LaTeX within your code
  escapechar={!},
  % extendedchars=true,              % lets you use non-ASCII characters; for 8-bits encodings only, does not work with UTF-8
  % firstnumber=1000,                % start line enumeration with line 1000
  % frame=tb,                        % adds a frame around the code
  % keepspaces=true,                 % keeps spaces in text, useful for keeping indentation of code (possibly needs columns=flexible)
  keywordstyle=\color{darkblue},     % keyword style
  language=Haskell,                  % the language of the code
  morekeywords={ Set, Tree, Leaf, Node, Applicative, fmap, liftA2, bimap, foldMap
               , traverse, mappend, pure, Foldable, Traversable, zero, one
               , isZero, Semiring, Semigroup, NonEmpty, sconcat, TSet,
               , SimplicialSet, TSimplicialSet, Graph, TGraph, LGraph
               , Map, IsString, fromString },
  deletekeywords={instance, data, where, class, filter, type, insert, delete, union, map},      % if you want to delete keywords from the given language
  emph={data, class, instance, where, type},
  emphstyle=\color{darkpink},
  numbers=none,                      % where to put the line-numbers; possible values are (none, left, right)
  % numbersep=5pt,                   % how far the line-numbers are from the code
  % numberstyle=\tiny\color{mygray}, % the style that is used for the line-numbers
  % rulecolor=\color{black},         % if not set, the frame-color may be changed on line-breaks within not-black text (e.g. comments (green here))
  % showspaces=false,                % show spaces everywhere adding particular underscores; it overrides 'showstringspaces'
  % showstringspaces=false,          % underline spaces within strings only
  % showtabs=false,                  % show tabs within strings adding particular underscores
  % stepnumber=2,                    % the step between two line-numbers. If it's 1, each line will be numbered
  stringstyle=\color{grayred},     % string literal style
  % tabsize=2,                       % sets default tabsize to 2 spaces
  % title=\lstname                   % show the filename of files included with \lstinputlisting; also try caption instead of title
  xleftmargin=10pt,
  aboveskip=8pt,
  belowskip=4pt
}

\newcommand{\code}[1]{\lstinline[mathescape]|#1|}
\newcommand{\hcode}[1]{{\color{darkblue} \lstinline[keywordstyle={}]|#1|}} % h for "highlighted"
\newcommand{\h}[1]{{\itshape\color{grayblue}#1}} % h for "highlighted"
\newcommand{\n}[1]{{\itshape\color{graygreen}#1}} % n for "normal"
\newcommand{\up}[1]{{\itshape\color{graygreen}\textsuperscript{#1}}}
\newcommand{\hadd}{{\large\color{darkblue} $\oplus$}}
\newcommand{\hmul}{{\large\color{darkblue} $\otimes$}}
\newcommand{\hdia}{\,\text{\raisebox{-0.2mm}{\Large\color{darkblue} $\diamond$}}\,}
\newcommand{\harr}{\,\text{\color{darkblue} $\rightarrow$}\,}

\newcommand{\add}{\text{\raisebox{-0.1mm}{\large $\oplus$}}}
\newcommand{\mul}{\text{\raisebox{-0.1mm}{\large $\otimes$}}}
\newcommand{\zero}{\raisebox{-0.2mm}{\textcircled{0}}\xspace}
\newcommand{\one}{\raisebox{-0.2mm}{\textcircled{1}}\xspace}
\newcommand{\two}{\raisebox{-0.2mm}{\textcircled{2}}\xspace}
\newcommand{\dia}{\,\text{\raisebox{-0.3mm}{\Large $\diamond$}}\,}
\newcommand{\arr}{\,\text{$\rightarrow$}\,}
\newcommand{\ldia}{\text{\raisebox{-0.3mm}{\Large $\diamond$}}}


\begin{document}

\paperdetails{
    perspective=sciencetheoretical,
    area={Functional Data Structures}
}

\title{United Monoids}
\subtitle{Finding Simplicial Sets and Labelled Algebraic Graphs on Trees}
\author{Andrey Mokhov}
\affiliation{Jane Street}

\keywords{algebra, graphs, functional programming}

\begin{CCSXML}
<ccs2012>
<concept>
<concept_id>10002950.10003624.10003633</concept_id>
<concept_desc>Mathematics of computing~Graph theory</concept_desc>
<concept_significance>500</concept_significance>
</concept>
</ccs2012>
\end{CCSXML}

\ccsdesc[500]{Mathematics of computing~Graph theory}

\maketitle

\begin{abstract}
This paper demonstrates how a simple inductive data type can be used as a
language for describing and manipulating various combinatorial graph-like
structures. By changing the rules of interpretation of the language, we will
obtain sets, preorders, edge-labelled graphs, and simplicial sets.

As a theoretical contribution, the paper presents united monoids, an algebraic
structure that turns out to be a common ground for the graph-like structures
discussed in the paper.
\end{abstract}

\section{Introduction}
\vspace{-1mm}

Graphs and various graph-like combinatorial structures are ubiquitous in
programming. There are numerous approaches to representing graphs, from good old
adjacency matrices~\cite{cormen2009introduction} to categorical graph
algebras~\cite{2010_selinger_survey}. This paper introduces a new purely
functional graph representation. It is built on elementary foundations
(trees and semirings), and provides an expressive language for modelling
graph-like structures.

The following data type of binary trees with \hcode{a}-labelled leaves and
\hcode{s}-labelled internal nodes will be the main protagonist of our
story\footnote{
    We will use Haskell throughout this paper but the presented ideas
    are not Haskell-specific and can be readily translated to other languages.
}:

\begin{lstlisting}
data Tree s a = Leaf a | Node s (Tree s a) (Tree s a)
\end{lstlisting}

\noindent
In this paper we are going to interpret such trees in a few different ways. To
do that, we will \emph{fold} them~\cite{gibbons_folds} by specifying different
pairs of functions \hcode{leaf} and \hcode{node}, which will give meanings (of
type \hcode{m}) to the leaves and internal nodes, respectively:

\begin{lstlisting}
fold :: (a -> m) -> (s -> m -> m -> m) -> Tree s a -> m
fold leaf node = meaning
  where
    meaning (Leaf a) = leaf a
    meaning (Node s x y) = node s (meaning x) (meaning y)
\end{lstlisting}

\noindent
For example, to compute the \emph{size} of a tree, i.e. the number of leaves in
it, we can fold the tree into an \code{Int} as follows.

\begin{lstlisting}
size :: Tree s a -> Int
size = fold (const 1) (const (+))
\end{lstlisting}

\noindent
Here we ignore the labels of leaves and internal nodes: every leaf is
interpreted as~$1$, and every internal node simply adds up the sizes of its
children. In the rest of the paper, we will study much more interesting ways to
fold trees. In particular, we will see how the very same data type can serve us
when working with sets, preorders, various flavours of graphs, as well as
simplicial sets. More specifically, the contributions of this paper are as
follows.

\begin{itemize}
  \item We extend \emph{algebraic graphs}~\cite{mokhov_alga} to support edge
  labels (\S\ref{sec-labelled}), and also simplify their inductive definition
  from four to just two constructors (the \code{Tree} data type,
  \S\ref{sec-tree}).

  \item To demonstrate the flexibility of the new approach, we show how it can
  be used to work with a variety of combinatorial structures, including
  simplicial sets and preorders (\S\ref{sec-set}-\S\ref{sec-preorder}). All of
  these seemingly different structures turn out to be just different
  interpretations of the same underlying language of trees.

  \item We introduce \emph{united monoids} as an algebraic structure that
  captures the essence of the various studied interpretations of trees
  (\S\ref{sec-united}). We also characterise united monoids with a minimal set
  of axioms.
\end{itemize}

\noindent
We review related work in~\S\ref{sec-related-work}.

\section{Trees}\label{sec-tree}

In this section we study the \hcode{Tree s a} data type in more detail. An
experienced functional programmer will find the data type and various
associated functions fairly standard, so this section can be skipped after
glancing through the definitions in Listings~\ref{lst-tree-std}
and~\ref{lst-tree}.

Throughout the paper, leaves of our trees will represent \emph{elements} (of
sets and preorders) or \emph{vertices} (of simplicial sets and graphs). We will
treat their labels~\hcode{a} as abstract, apart from the occasional requirement
of \hcode{Eq a} or \hcode{Ord a}, e.g., to be able to collect leaves into a set.
Internal nodes, on the other hand, will represent various kinds of
\emph{connectivity}, and we will require that their labels~\hcode{s} come from
a semiring~\cite{1999_semirings_golan}.

A \emph{semiring} is an algebraic structure that generalises arithmetic: it's a
set equipped with associative\footnote{
    An operation $\bullet$ is \emph{associative} if
    $(a \bullet b) \bullet c = a \bullet (b \bullet c)$. Associativity is
    convenient both for humans (making parentheses unnecessary) and machines
    (allowing large expressions to be efficiently processed by partitioning them
    into smaller subexpressions as necessary).
}
operations of \emph{addition}~$\add$ and \emph{multiplication}~$\mul$, whose
units\footnote{
    An element $e$ is the \emph{unit} (or the \emph{identity element}) of an
    operation $\bullet$ if $e \bullet a = a \bullet e = a$.
}
are \emph{zero}~$\text{\zero}$ and \emph{one}~$\text{\one}$, respectively.
Semirings give us a basic language for connectivity:

\vspace{-1mm}
\begin{itemize}
  \item The lack of any connectivity can be expressed by~\zero; e.g, zero
        network bandwidth.
  \item Addition~$\add$ is a commutative\footnote{
            An operation $\bullet$ is \emph{commutative} if
            $a \bullet b = b \bullet a$.
        } operation for combining connections in \emph{parallel}; for example,
        adding bandwidth of independent network links between two locations.
  \item Multiplication~$\mul$ is for combining connections in \emph{sequence}.
        Continuing our example, $\mul$ corresponds to taking the \emph{minimum}
        bandwidth of network links on route between two locations.
        Multiplication distributes\footnote{
            \emph{Distributivity} of $\otimes$ over $\oplus$ means
            $a \otimes (b \oplus c) = (a \otimes b) \oplus (a \otimes c)$ and
            $(a \oplus b) \otimes c = (a \otimes c) \oplus (b \otimes c)$.
        } over addition and has \zero as a zero\footnote{
            An element $z$ is a \emph{zero} (or an \emph{annihilating element})
            of an operation $\bullet$ if $z \bullet a = a \bullet z = z$.
        }.
  \item Finally, \one is used to model either a unit of connectivity or infinite
        connectivity, depending on the specific semiring. In the bandwidth
        semiring, we have $\text{\one} = \infty$.
\end{itemize}

\noindent
We first turn our attention to trees whose internal nodes are labelled
with~\zero, i.e., whose leaves are \emph{disconnected}. For example, the trees
shown below correspond to expressions \code{"a"}, \code{"b" $\,\hdia\,$ "b"},
\code{("a" $\,\hdia\,$ "b") $\,\hdia\,$ "c"} and
\code{"a" $\,\hdia\,$ ("b" $\,\hdia\,$ "c")}, where the operator \code{$\dia$}
comes from the \code{Semigroup (Tree s a)} instance, and string literals are
desugared into leaves using the \code{OverloadedStrings} extension
(see Listing~\ref{lst-tree-std}).

\vspace{2.2mm}
\hfill\includesvg[scale=0.28]{tree-examples}\hfill
\vspace{2.3mm}

\noindent
A \emph{semigroup} is a set with an associative operation. It may seem worrying
that our \code{Semigroup} instance yields different trees for left- and
right-associated expressions, as in the last two examples. To resolve this, we
make the \code{Tree} data type abstract and require all uses of \hcode{fold} to
respect the associativity when interpreting trees. Concretely, the
\hcode{fold}'s argument \hcode{node} must yield associative functions for all
\hcode{s}, including \hcode{s = zero}.

\begin{lstlisting}[float,label=lst-tree-std,xleftmargin=0pt,caption={
    The \code{Tree} data type and instances of various standard Haskell type classes.
}]
!\hrulefill!
data Tree s a = Leaf a | Node s (Tree s a) (Tree s a)

fold :: (a -> m) -> (s -> m -> m -> m) -> Tree s a -> m -- Note: !\h{node s}! must be associative
fold leaf node = meaning
  where
    meaning (Leaf a) = leaf a
    meaning (Node s x y) = node s (meaning x) (meaning y)

class Semiring s where -- In this paper, !\h{s}! in !\h{Tree s a}! is always a !\h{Semiring}!
    zero !\hspace{2.9mm}!:: s !\hspace{14.9mm}!-- Unit of !\hadd!, zero of !\hmul!
    isZero :: s -> Bool !\hspace{2.5mm}!-- Testing for zero is often easier than implementing !\h{Eq s}!
    (!\hadd!) !\hspace{4.45mm}!:: s -> s -> s !\hspace{1.4mm}!-- Associative, commutative
    one !\hspace{3.65mm}!:: s !\hspace{14.9mm}!-- Unit of !\hmul!
    (!\hmul!) !\hspace{4.45mm}!:: s -> s -> s !\hspace{1.4mm}!-- Associative, distributes over !\hadd!

instance Semiring s => Semigroup (Tree s a) where
    (!\hdia!) :: Tree s a -> Tree s a -> Tree s a
    (!\hdia!) = Node zero -- All tree interpretations must respect the associativity of !\hdia!

instance Functor (Tree s) where
    fmap :: (a -> b) -> Tree s a -> Tree s b -- Apply a function !\h{a -> b}! to every leaf !\h{a}!
    fmap f = fold (Leaf . f) Node

instance Applicative (Tree s) where
    pure :: a -> Tree s a -- Create a !\n{``}!trivial!\n{''}! tree containing a single leaf
    pure = Leaf
    (<*>) :: Tree s (a -> b) -> Tree s a -> Tree s b -- Graft the 2!\up{nd}! tree on every leaf of the 1!\up{st}!
    (<*>) = ap !\hspace{52.6mm}!-- Standard implementation via Monad

instance Monad (Tree s) where
    (>>=) :: Tree s a -> (a -> Tree s b) -> Tree s b -- Graft a tree !\h{f a}! on every leaf !\h{a}!
    x >>= f = fold f Node x

instance Foldable (Tree s) where
    foldr :: (a -> b -> b) -> b -> Tree s a -> b -- Fold leaves from right to left
    foldr f b tree = fold f (const (.)) tree b

instance Traversable (Tree s) where
    traverse :: Applicative f => (a -> f b) -> Tree s a -> f (Tree s b) -- Effectful variant of !\h{fmap}!
    traverse f = fold (fmap Leaf . f) (liftA2 . Node)

instance IsString a => IsString (Tree s a) where -- This makes !\h{"a"}! a shortcut for !\h{Leaf "a"}!
    fromString = Leaf . fromString
!\rule[2mm]{\textwidth}{0.4pt}!
\end{lstlisting}

\begin{lstlisting}[float,label=lst-tree,xleftmargin=0pt,belowskip=-2mm,caption={
    API for constructing and manipulating trees.
}]
!\rule[2mm]{\textwidth}{0.4pt}!
!\vspace{-7mm}!
leaf :: a -> Tree s a -- A tree with a single leaf
leaf = Leaf
!\vspace{-1mm}!
leaves :: Semiring s => NonEmpty a -> Tree s a -- A tree of !\n{``}!disconnected!\n{''}! leaves
leaves = sconcat . NonEmpty.map Leaf -- Via !\h{Semigroup}!'s !\h{sconcat}!
!\vspace{-1mm}!
node :: s -> Tree s a -> Tree s a -> Tree s a -- Combine two trees into an !\h{s}!-labelled node
node = Node
!\vspace{-1mm}!
size :: Tree s a -> Int -- The number of leaves in a tree
size = length -- Via !\h{Foldable}!
!\vspace{-1mm}!
leafSet :: Ord a => Tree s a -> Set a -- The set of leaves of a tree
leafSet = fold Set.singleton (const Set.union)
!\vspace{-1mm}!
hasLeaf :: Eq a => a -> Tree s a -> Bool -- Test if a tree contains a leaf !\h{a}!
hasLeaf = elem -- Via !\h{Foldable}!
!\vspace{-1mm}!
prune :: Tree s (Maybe a) -> Maybe (Tree s a) -- Prune all leaves labelled with !\h{Nothing}!
prune = sequence -- Via !\h{Traversable}!
!\vspace{-1mm}!
filter :: (a -> Bool) -> Tree s a -> Maybe (Tree s a) -- Prune all leaves !\h{a}! with !\h{p a = False}!
filter p = traverse (\a -> if p a then Just a else Nothing) -- Via !\h{Traversable}!
!\rule[2mm]{\textwidth}{0.4pt}!
\end{lstlisting}

\noindent
It is sometimes necessary to be able to express the \emph{empty tree}. To do
that, we simply wrap the \hcode{Tree} data type into \hcode{Maybe}. This is
illustrated by the function \hcode{prune} (Listing~\ref{lst-tree}), which prunes
all leaves labelled with \hcode{Nothing} and returns a possibly empty tree as a
result. By combining the empty tree and the operation \dia, we can extend our
type of trees from a semigroup to a \emph{monoid}, i.e., a semigroup with the
unit element.

When tree interpretations respect the associativity of \dia, \zero-labelled
trees become isomorphic to non-empty lists. This is evidenced by the functions
\hcode{toList} (from the \hcode{Foldable} type class in
Listing~\ref{lst-tree-std}) and \hcode{leaves} (Listing~\ref{lst-tree}). In the
subsequent sections, we will study data structures obtained by instantiating
\hcode{Tree s a} with various semirings \hcode{s}, and by adding more
requirements (on top of associativity) to tree interpretations.

\section{Sets}\label{sec-set}

In this section we instantiate \hcode{Tree s a} with the \emph{trivial semiring}
\hcode{s = ()} where $\zero = \one = \text{\hcode{()}}$:

\begin{lstlisting}
type TSet a = Tree () a -- We prepend !\h{T}! to avoid a name clash with the standard !\h{Set}!
\end{lstlisting}

\noindent
As hinted by the type synonym's name, we will interpret values \hcode{TSet a} as
(non-empty) sets; to get a standard \hcode{Set a} from a \hcode{TSet a}, we can
simply reuse \hcode{leafSet} from Listing~\ref{lst-tree}.
In fact, as demonstrated in Listing~\ref{lst-set}, a large part of the standard
set API can be be obtained by reusing more general \hcode{Tree}-manipulating
functions defined earlier in~\S\ref{sec-tree}.

\begin{lstlisting}[label=lst-set,xleftmargin=0pt,belowskip=7pt,caption={
    Implementing a part of the standard \code{Data.S}\code{et} API with
    \hcode{TSet}.
}]
!\rule[2mm]{\textwidth}{0.4pt}!
!\vspace{-7mm}!
type TSet a = Tree () a -- Sets are trees with !\h{()}!-labelled internal nodes
!\vspace{-1mm}!
instance Semiring () where -- The trivial semiring with !\h{zero = one = ()}!
    zero !\hspace{6.1mm}!= ()
    isZero () = True
    () !\add! () !\hspace{3.55mm}!= ()
    one !\hspace{7mm}!= ()
    () !\mul! () !\hspace{3.55mm}!= ()
!\vspace{-1mm}!
singleton :: a -> TSet a !\hspace{32.18mm}\vrule\hspace{8mm}!insert :: a -> TSet a -> TSet a
singleton = Tree.leaf    !\hspace{35.1mm}\vrule\hspace{8mm}!insert = (!\hdia!) . singleton -- Via !\h{Semigroup}!
!\vspace{-1mm}!
fromList :: [a] -> Maybe (TSet a) -- The empty set is represented by !\h{Nothing}!
fromList = fmap Tree.leaves . NonEmpty.nonEmpty
!\vspace{-1mm}!
delete :: Eq a => a -> TSet a -> Maybe (TSet a) -- The result can be empty
delete a = Tree.filter (/= a)
!\vspace{-1mm}!
member :: Eq a => a -> TSet a -> Bool !\hspace{9.1mm}\vrule\hspace{8mm}!union :: TSet a -> TSet a -> TSet a
member = Tree.hasLeaf                 !\hspace{30.33mm}\vrule\hspace{8mm}!union = (!\hdia!) -- Via !\h{Semigroup}!
!\vspace{-1mm}!
size :: Ord a => TSet a -> Int
size = Set.size . leafSet -- We can't use !\h{Tree.size}! because it uses a non-idempotent !\h{fold}!
!\vspace{-1mm}!
cartesianProduct :: TSet a -> TSet b -> TSet (a, b)
cartesianProduct = liftA2 (,) -- Via !\h{Applicative}!
!\vspace{-1mm}!
filter :: (a -> Bool) -> TSet a -> Maybe (TSet a) -- The result can be empty
filter = Tree.filter
!\vspace{-1mm}!
map :: (a -> b) -> TSet a -> TSet b
map = fmap -- Via !\h{Functor}!
!\rule[2mm]{\textwidth}{0.4pt}!
\end{lstlisting}

\noindent
In addition to associativity, the set interpretation of trees puts two new
requirements on every call site \hcode{fold leaf node}\footnote{
    We should use \hcode{fold} with care since Haskell's type system isn't
    powerful enough to check such requirements. A dependently typed language,
    such as Agda~\cite{2007_norell_agda}, could help us here.
}:

\begin{itemize}
    \item $\text{\hcode{node zero a b}} = \text{\hcode{node zero b a}}$, i.e.,
          commutativity of \zero-labelled nodes;
    \item $\text{\hcode{node zero a a}} = \text{\hcode{a}}$, i.e.,
    idempotence\footnote{
         An operation $\bullet$ is \emph{idempotent} if $a \bullet a = a$.
    } of \zero-labelled nodes.
\end{itemize}

\noindent
Therefore we can't use \hcode{Tree.}\code{size} in the implementation of
\hcode{TSet.}\code{size} in Listing~\ref{lst-set}: \hcode{Tree.}\code{size} uses
integer addition for folding internal nodes, which is not idempotent. And
indeed, \hcode{Tree.}\code{size ("b" $\,\hdia\,$ "b")}$\ =2$ since the tree has
\emph{two leaves}, whereas \hcode{TSet.}\code{size ("b" $\,\hdia\,$ "b")}$\ = 1$
since the set described by the tree has only \emph{one element}.

\section{Simplicial sets}\label{sec-simplicial-set}

Describing plain sets with trees isn't new or particularly exciting. However, we
hope the reader found it instructive to see what happens if we instantiate
\hcode{Tree s a} with the simplest semiring \hcode{s = ()}. This section
continues by trying the next simplest semiring, i.e., the
\emph{Boolean semiring} \hcode{s = Bool}, where $\zero = \text{\code{False}}$,
$\one = \text{\code{True}}$, $\add = \text{\code{(\|\|)}}$ and
$\mul = \text{\code{(&&)}}$.

A \emph{simplicial set}~\cite{friedman_simplicial_sets} is a set of
\emph{simplices} of various dimensions along with their incidence relation.
A \emph{0-simplex} is a just a point, or a \emph{vertex}, and it will correspond
to a leaf in our trees. A \emph{1-simplex} is a pair of connected vertices, or
an \emph{edge}; a \emph{2-simplex} is a filled-in \emph{triangle}; a
\emph{3-simplex} is a solid \emph{tetrahedron}; and so on. To represent
simplices of dimension higher than $0$, we will \emph{connect} their vertices
using \one-labelled internal nodes. The figure below shows the simplicial sets
corresponding to the trees \code{"a"}, \code{"a" $\,\harr\,$ "b"},
\code{("a" $\,\harr\,$ "b") $\,\harr\,$ "c"} and
\code{("a" $\,\hdia\,$ "b") $\,\harr\,$ "c"}, where
\code{($\harr$)}~\code{=}~\hcode{node one} (see Listing~\ref{lst-simplicial-set}).

\vspace{5mm}
\hfill\hspace{-4mm}\includesvg[scale=0.28]{simplicial-set-examples}\hfill
\vspace{5mm}

\noindent
Note the difference between the last two examples: by changing the label of one
of the nodes from \one to \zero, we break the corresponding connection,
and as a result, a 2-simplex (triangle $\textit{abc}$) falls apart into
one-dimensional simplices (edges $\textit{ac}$~and~$\textit{bc}$).

When working with simplicial sets, we will extend the requirements on
interpreting trees with \hcode{fold} by adding the distributivity requirement:

\begin{itemize}
    \item Associativity, inherited from~\S\ref{sec-tree}:
          $(a \dia b) \dia c = a \dia (b \dia c)$,
          $(a \arr b) \arr c = a \arr (b \arr c)$;
    \item Commutativity, inherited from~\S\ref{sec-set}:
          $a \dia b = b \dia a$;
    \item Idempotence, inherited from~\S\ref{sec-set}:
          $a \dia a = a$;
    \item Distributivity (\textbf{\color{darkblue}new!}):
          $a \arr (b \dia c) = (a \arr b) \dia (a \arr c)$,
          $(a \dia b) \arr c = (a \arr c) \dia (b \arr c)$.
\end{itemize}

\noindent
The new requirement is motivated by the desire to make the trees
\code{("a" $\,\hdia\,$ "b") $\,\harr\,$ "c"} and
\code{("a" $\,\harr\,$ "b") $\,\hdia\,$ ("a" $\,\harr\,$ "c")} mean the same
simplicial set (the rightmost one above).

Now that we know the rules of interpreting trees over the Boolean semiring, we
need to choose a target representation for simplicial sets. Alas, there is no
standard \code{Data.SimplicialS}\code{et}, so we need to come up with our own.
Here is a simple candidate:

\begin{lstlisting}
type SimplicialSet a = Set (NonEmpty a)
\end{lstlisting}

\noindent
Here the inner non-empty lists correspond to simplices, with their vertices
listed in the connection order. For example, the simplicial set corresponding to
the tree \code{"a" $\,\harr\,$ "b"} will contain three lists: \code{[a]},
\code{[b]} and \code{[a,b]}.

\begin{lstlisting}[float,label=lst-simplicial-set,xleftmargin=0pt,caption={
    A basic API for working with simplicial sets represented by trees.
}]
!\rule[2mm]{\textwidth}{0.4pt}!
!\vspace{-7mm}!
type TSimplicialSet a = Tree Bool a -- Simplicial sets are trees over the Boolean semiring
!\vspace{-1mm}!
instance Semiring Bool where -- The Boolean semiring with !\h{zero = False}! and !\h{one = True}!
    zero !\hspace{2.95mm}!= False
    isZero = not
    (!\hadd!) !\hspace{4.45mm}!= (||)
    one !\hspace{3.65mm}!= True
    (!\hmul!) !\hspace{4.45mm}!= (&&)
!\vspace{-1mm}!
(!\harr!) :: TSimplicialSet a -> TSimplicialSet a -> TSimplicialSet a
(!\harr!) = Tree.node one
!\vspace{-1mm}!
overlay :: TSimplicialSet a -> TSimplicialSet a -> TSimplicialSet a
overlay = (!\hdia!) -- Via !\h{Semigroup}!; recall that !\h{(\hdia) = Tree.node zero}!
!\vspace{-1mm}!
connect :: TSimplicialSet a -> TSimplicialSet a -> TSimplicialSet a
connect = (!\harr!)
!\vspace{-1mm}!
vertex :: a -> TSimplicialSet a -- 0-simplex
vertex = Tree.leaf
!\vspace{-1mm}!
edge :: a -> a -> TSimplicialSet a -- 1-simplex
edge x y = connect (vertex x) (vertex y)
!\vspace{-1mm}!
triangle :: a -> a -> a -> TSimplicialSet a -- 2-simplex
triangle x y z = connect (vertex x) (edge y z)
!\vspace{-1mm}!
simplex :: NonEmpty a -> TSimplicialSet a -- N-simplex
simplex = foldr1 connect . NonEmpty.map vertex -- Via !\h{Foldable}!
!\vspace{-1mm}!
hasVertex :: Eq a => a -> TSimplicialSet a -> Bool
hasVertex = Tree.hasLeaf
!\vspace{-1mm}!
vertexSet :: Ord a => TSimplicialSet a -> Set a
vertexSet = Tree.leafSet
!\vspace{-1mm}!
filter :: (a -> Bool) -> TSimplicialSet a -> Maybe (TSimplicialSet a)
filter = Tree.filter
!\vspace{-1mm}!
map :: (a -> b) -> TSimplicialSet a -> TSimplicialSet b -- The so-called !\n{``}!simplicial map!\n{''}!
map = fmap -- Via !\h{Functor}!
!\rule[2mm]{\textwidth}{0.4pt}!
\end{lstlisting}

\noindent
Now we can define \hcode{vertex}, \hcode{overlay} and \hcode{connect} to
interpret leaves, as well as \zero-~and \one-labelled nodes, respectively, and
use them to fold a \hcode{Tree Bool a} into a \hcode{SimplicialSet a}:

\begin{lstlisting}
vertex :: a -> SimplicialSet a
vertex = Set.singleton . NonEmpty.singleton
!\vspace{-1mm}!
overlay :: Ord a => SimplicialSet a -> SimplicialSet a -> SimplicialSet a
overlay = Set.union
!\vspace{-1mm}!
connect :: Ord a => SimplicialSet a -> SimplicialSet a -> SimplicialSet a
connect x y = Set.unions [x, y, Set.map (uncurry (!\hdia!)) (Set.cartesianProduct x y)]
!\vspace{-1mm}!
toSimplicialSet :: Ord a => Tree Bool a -> SimplicialSet a
toSimplicialSet = fold vertex (bool overlay connect)
\end{lstlisting}

\noindent
Note that \hcode{SimplicialSet} is a rather naive representation for simplicial
sets. In particular, an $n$-simplex contains $O(2^n)$ sub-simplices of
lower dimension, and all of them will be included in the outer \hcode{Set}. It
is possible to improve the representation by storing only maximal simplices, or
by using suffix trees~\cite{weiner_suffix_trees}. The \hcode{Tree}-based
representation of simplicial sets is free from this problem: an $n$-simplex can
be described by an expression of size $O(n)$, as evidenced by the function
\hcode{simplex} in Listing~\ref{lst-simplicial-set}.

Another advantage of the \hcode{Tree}-based representation is that it is free
from internal invariants, i.e. any value of \hcode{Tree Bool a} describes a
valid simplicial set. \hcode{SimplicialSet} does not satisfy this property; for
example, the set containing lists \code{[a]} and \code{[a,b]} is not a
simplicial set, because the edge $\textit{ab}$ appears in the set without its
vertex $\textit{b}$.

\vspace{-2mm}
\section{Graphs}\label{sec-graph}

Graphs can be thought of as simplicial sets that contain only 0-simplices
(vertices) and 1-simplices (edges). One can therefore represent graphs with a
pair of sets:

\begin{lstlisting}
type Graph a = (Set a, Set (a, a)) -- A set of vertices and a set of edges
\end{lstlisting}

\noindent
To interpret trees \hcode{Tree Bool a} as graphs, we need a way to break up
simplices of dimension greater than $1$ into edges. To do that, we impose a new
folding requirement:

\begin{itemize}
    \item Decomposition:
          $a \arr b \arr c = (a \arr b) \dia (a \arr c) \dia (b \arr c)$.
\end{itemize}

\noindent
This law is unusual but it is not new: it was introduced as an axiom of
\emph{algebraic graphs} in~\cite{mokhov_alga}. In fact, by adding this
requirement to those used when interpreting simplicial sets
in~\S\ref{sec-simplicial-set}, we get exactly the algebra of non-empty graphs
from~\cite{mokhov_alga}. It is therefore not surprising that we can reuse a lot
of more general functions operating on trees and simplicial sets when
implementing a part of the API of algebraic graphs -- see
Listing~\ref{lst-graph}, where \hcode{hasEdge} is the only function that we did
not reuse.

Let us now check that we can fold a \hcode{Tree Bool a} into the corresponding
\hcode{Graph a} while respecting decomposition. The last step will be identical
to the one from~\S\ref{sec-simplicial-set}:

\begin{lstlisting}
toGraph :: Ord a => TGraph a -> Graph a
toGraph = fold vertex (bool overlay connect)
\end{lstlisting}

\noindent
All we need to do is adapt \hcode{vertex}, \hcode{overlay} and \hcode{connect}
to ``truncated'' simplicial sets. The ``truncated'' variants of \hcode{vertex}
and \hcode{overlay} are straightforward:

\begin{lstlisting}
vertex :: a -> Graph a
vertex a = (Set.singleton a, Set.empty)

overlay :: Ord a => Graph a -> Graph a -> Graph a
overlay (v1, e1) (v2, e2) = (Set.union v1 v2, Set.union e1 e2)
\end{lstlisting}

\noindent
To connect two graphs, we augment their original edges with a Cartesian product
of their vertex sets:

\begin{lstlisting}
connect :: Ord a => Graph a -> Graph a -> Graph a
connect (v1,e1) (v2,e2) = (Set.union v1 v2, Set.unions [e1, e2, Set.cartesianProduct v1 v2])
\end{lstlisting}

\noindent
It is not obvious that this interpretation respects the decomposition
requirement, so let us check it. For the vertex set, the requirement holds
because $\cup$ is idempotent:

\vspace{-5mm}
\begin{equation*}
v_1 \cup v_2 \cup v_3 = (v_1 \cup v_2) \cup (v_1 \cup v_3) \cup (v_2 \cup v_3)
\end{equation*}
\vspace{-5mm}

\noindent
As for the edge set, one can mechanically check that both sides of the
decomposition requirement evaluate to the following symmetric expression.

\vspace{-5mm}
\begin{equation*}
e_1 \cup e_2 \cup e_3 \cup (v_1 \times v_2) \cup (v_1 \times v_3) \cup (v_2 \times v_3)
\end{equation*}
\vspace{-5mm}

\subsection{Other kinds of graphs}

So far we discussed only directed graphs with no edge labels. One way to support
other kinds of graphs is to strengthen the tree folding requirements.
Specifically:

\begin{itemize}
    \item For \emph{undirected graphs}, the operator $\arr$ needs to be
    commutative, i.e. $a \arr b = b \arr a$.
    \item For \emph{reflexive graphs}, the operator $\arr$ needs to be
    idempotent with respect to single vertices, i.e.
    \hcode{vertex a}$\arr$\hcode{vertex a}~$=$~\hcode{vertex a}.
    \item For \emph{bipartite graphs}, the operator $\arr$ should take no
    effect on vertices from the same part, i.e.
    $\forall a,b \in (P \times P) \cup (Q \times Q),\ a \arr b = a \dia b$,
    where $P$ and $Q$ are the two parts of the bipartite graph.
    \item For \emph{transitively-closed graphs}, see \S\ref{sec-preorder}.
\end{itemize}

\noindent
Another approach to support new kinds of graphs is to instantiate
\hcode{Tree s a} with other semirings \hcode{s}. In~\S\ref{sec-labelled}, this
will allow us to support \emph{edge-labelled graphs}.

\begin{lstlisting}[float,label=lst-graph,xleftmargin=0pt,caption={
    Implementing a part of the \code{Algebra.G}\code{raph} API~\cite{mokhov_alga}
    with \hcode{TGraph}.
}]
!\rule[2mm]{\textwidth}{0.4pt}!
!\vspace{-7mm}!
type TGraph a = Tree Bool a -- Plus the !\h{Functor}!, !\h{Applicative}! and !\h{Monad}! instances

vertex :: a -> TGraph a
vertex = SimplicialSet.vertex

edge :: a -> a -> TGraph a
edge = SimplicialSet.edge

vertices :: NonEmpty a -> TGraph a
vertices = Tree.leaves

clique :: NonEmpty a -> TGraph a -- A clique is a set of pairwise connected vertices
clique = SimplicialSet.simplex

overlay :: TGraph a -> TGraph a -> TGraph a
overlay = SimplicialSet.overlay

connect :: TGraph a -> TGraph a -> TGraph a
connect = SimplicialSet.connect

vertexSet :: Ord a => TGraph a -> Set a
vertexSet = SimplicialSet.vertexSet

edgeSet :: Ord a => TGraph a -> Set (a, a)
edgeSet = snd . toGraph

induce :: (a -> Bool) -> TGraph a -> Maybe (TGraph a) -- The result may be empty
induce = Tree.filter

hasVertex :: Eq a => a -> TGraph a -> Bool
hasVertex = SimplicialSet.hasVertex

hasEdge :: Ord a => a -> a -> TGraph a -> Bool --
hasEdge x y = maybe False (elem (x,y) . edgeSet) . induce (\a -> a == x || a == y)
!\rule[2mm]{\textwidth}{0.4pt}!
\end{lstlisting}

\section{Preorders}\label{sec-preorder}

A \emph{preorder} is a binary relation $\preceq$ that is:

\begin{itemize}
    \item \emph{Reflexive}: $a \preceq a$; and
    \item \emph{Transitive}: $a \preceq b \wedge b \preceq c \Rightarrow a \preceq c$.
\end{itemize}

\noindent
In this section we will see how to interpret trees as preorders.

\newpage
\noindent
We could build directly on top of the infrastructure developed for graphs
in~\S\ref{sec-graph}, but instead we will generalise our approach and switch to
a setting with general semirings, i.e., trees of type \hcode{Tree s a} for an
arbitrary semiring \hcode{s}. This leads us to the following encoding of the
reflexivity and transitivity requirements:

\begin{itemize}
    \item Reflexivity: \hcode{node one (leaf a) (leaf a)}~$=$~\hcode{leaf a}. In
    words, connecting a leaf to itself with a \one-labelled node is redundant.
    \item Transitivity:
    \hcode{node x a b}~\hdia~\hcode{node y b c}~\hdia~\hcode{node (x}~\hmul~\hcode{y) a c}~$=$~\hcode{node x a b}~\hdia~\hcode{node y b c}.
    This ensures that we interpret trees modulo transitivity, i.e., adding or
    removing a transitive edge \hcode{node (x}~\hmul~\hcode{y) a c} to a tree
    expression should not matter.
\end{itemize}

\noindent
To make expressions involving trees over general semirings clearer, we introduce
the following notation:

\vspace{-5mm}
\begin{equation*}
a \xrightarrow{x} b = \text{\hcode{node x a b}}
\end{equation*}
\vspace{-5mm}

\noindent
With this notation, the above transitivity requirement can be expressed as

\vspace{-5mm}
\begin{equation*}
a \xrightarrow{x} b\ \dia\ b \xrightarrow{y} c\ \dia\ a \xrightarrow{x \mul y} c = a \xrightarrow{x} b\ \dia\ b \xrightarrow{y} c.
\end{equation*}
\vspace{-5mm}

\noindent
One can notice that this rule is reminiscent of the relaxation step in various
shortest-path finding algorithms, where a distance between two vertices $a$ and
$c$ is updated after discovering a transitive route via a vertex $b$ (recall
that in semirings~\mul~is used for sequential composition of distances).

One way to interpret trees and meet the reflexivity and transitivity
requirements is to use the Floyd-Warshall-Kleene
algorithm~\cite{hopcroft_ullman}\cite{kleene1951representation}. An
implementation is available in the Appendix.

\section{Edge-labelled graphs}\label{sec-labelled}

Armed with the notation from the previous section~\S\ref{sec-preorder}, we can
now use our little tree language to describe graphs whose edges are labelled
with values from an arbitrary semiring \hcode{s}. For example, the expression
$\text{\code{"a"}} \xrightarrow{\one} \text{\code{"b"}}\ \dia\ \text{\code{"b"}} \xrightarrow{\two} \text{\code{"c"}}$
will correspond to the graph below.

\vspace{2mm}
\hfill\includesvg[scale=0.28]{labelled-graph-example}\hfill
\vspace{3mm}

\noindent
To translate the shortcut notation to Haskell, we define the following two
operators:

\begin{lstlisting}
(-<) :: Tree s a -> s -> (Tree s a, s)
x -< s = (x, s)
!\vspace{-1mm}!
(>-) :: (Tree s a, s) -> Tree s a -> Tree s a
(x, s) >- y = node s x y
\end{lstlisting}

\noindent
This allows us to write the above example as
\code{"a"}~\code{-<1>-}~\code{"b"}~\hdia~\code{"b"}~\code{-<2>-}~\code{"c"}.

\newpage
\noindent
We will interpret such tree expressions into the following data type of
edge-labelled graphs, which maps vertices \hcode{a} to a collection of
\hcode{s}-labelled outgoing edges.

\begin{lstlisting}
type LGraph s a = Map a (Map a s)
\end{lstlisting}

\noindent
Creating singleton graphs and overlaying graphs is fairly straightforward:

\begin{lstlisting}
vertex :: a -> LGraph s a
vertex x = Map.singleton x Map.empty

overlay :: (Ord a, Semiring s) => LGraph s a -> LGraph s a -> LGraph s a
overlay = Map.unionWith (Map.unionWith (!\h{\hadd}!)) -- Combine parallel edge labels with !\h{\hadd}!
\end{lstlisting}

\noindent
When implementing \hcode{connect}, we come across a subtle problem with
\hcode{LGraph}: it makes it possible to add and accumulate redundant
\zero-labelled edges in the inner \hcode{Map}. This is something that the
\hcode{Tree}-based representation rules out.

\begin{lstlisting}
connect :: (Ord a, Semiring s) => s -> LGraph s a -> LGraph s a -> LGraph s a
connect s x y
    | isZero s  = overlay x y -- Avoid redundant !\h{zero}!-labelled edges
    | otherwise = Map.unionsWith (Map.unionWith (!\h{\hadd}!)) [ x, y, newEdges ]
  where
    outgoing = Map.fromSet (const s) (Map.keysSet y)
    newEdges = Map.fromSet (const outgoing) (Map.keysSet x)
\end{lstlisting}

\noindent
Finally, we can fold a \hcode{Tree s a} to turn it into the corresponding
\hcode{LGraph s a}:

\begin{lstlisting}
toLGraph :: (Ord a, Semiring s) => Tree s a -> LGraph s a
toLGraph = fold vertex connect
\end{lstlisting}

\noindent
It is useful to combine edge-labelled graphs with the reflexivity and
transitivity requirements described in the previous section on
preorders~(\S\ref{sec-preorder}). By computing the reflexive and transitive
closure of an edge-labelled graph, one can solve a variety of semiring
optimisation problems, e.g. see~\cite{2013_semirings_dolan}.

We have now seen how to use trees as a language for describing and manipulating
various kinds of sets and graphs. The next section introduces \emph{united
monoids} that provide a common ground for these seemingly different
combinatorial data structures.

\section{United Monoids}\label{sec-united}

In this section we introduce \emph{united monoids} as an algebraic structure
that turns out to be a common ground for the graph-like structures
discussed earlier in~\S\ref{lst-set}-\S\ref{sec-labelled}.

A \emph{monoid} $(S, \ldia, \varepsilon)$ is a way to express a basic form of
composition in mathematics: any two elements $a$ and $b$ of the set $S$ can be
composed into a new element $a \dia b$ of the same set $S$, and, furthermore,
there is a special element $\varepsilon \in S$, which is the unit element of the
composition, as expressed by the \emph{unit axioms}
$a \dia \varepsilon = \varepsilon \dia a = a$. In words, composing the unit
element with another element does not change the latter.

\noindent
Monoids often come in pairs: addition and multiplication $(+, \times)$,
disjunction and conjunction $(\vee, \wedge)$, set union and intersection
$(\cup, \cap)$, parallel and sequential composition of processes
$(\text{\code{|}},\ \text{\code{;}})$, etc. Two common ways in which such monoid
pairs can form are called \emph{semirings} and \emph{lattices}. In fact, the
former have played a major role in this paper so far. Below we briefly introduce
the latter.

A \emph{bounded lattice} $(S, \vee, 0, \wedge, 1)$ comprises two monoids, which
are called \emph{join} $(S, \vee, 0)$ and \emph{meet} $(S, \wedge, 1)$. They
operate on the same set, are required to be commutative and idempotent, and
satisfy the following \emph{absorption axioms}: $a \wedge (a \vee b) = a \vee (a \wedge b) = a$. Like semirings, lattices show up very frequently in different application
areas.

\subsection{What if $0=1$?}

What happens when the units of the two monoids in a pair coincide, i.e., when
$0=1$?
In a semiring $(S, \add, \zero, \mul, \one)$, this leads to devastating
consequences. Not only \one becomes equal to \zero, but all other elements of
the semiring become equal to \zero too, as demonstrated below.

\vspace{-5mm}
\begin{equation*}
\begin{array}{rcll}
a & = & \one\ \mul\ a & \text{(unit of $\mul$)}\\
 & = & \zero\ \mul\ a & \text{(we postulate \zero = \one)}\\
 & = & \zero & \text{(annihilating \zero)}
\end{array}
\end{equation*}
\vspace{-3mm}

\noindent
That is, the semiring is annihilated into a single point \zero, becoming
isomorphic to the trivial semiring \hcode{s = ()}, which we have come across
in~\S\ref{sec-set}.

In a bounded lattice $(S, \vee, 0, \wedge, 1)$, postulating $0 = 1$ leads to the
same catastrophe, albeit in a different manner:

\vspace{-5mm}
\begin{equation*}
\begin{array}{rcll}
a & = & 1\ \wedge\ a & \text{(unit of $\wedge$)}\\
 & = & 1\ \wedge\ (0\  \vee\  a) & \text{(unit of $\vee$)}\\
 & = & 0\ \wedge\ (0\  \vee\  a) & \text{(we postulate 0 = 1)}\\
 & = & 0 & \text{(absorption axiom)}
\end{array}
\end{equation*}
\vspace{-3mm}

\noindent
That is, the lattice is absorbed into a single point $0$.

Postulating $0 = 1$ has so far led to nothing but disappointment. In the
next subsection we find another way of pairing monoids, which does not involve
the axioms of annihilation and absorption, and makes the resulting structure
more interesting.

\subsection{From $0=1$ to containment laws}

Consider two monoids $(S, +, 0)$ and $(S, \cdot, 1)$, such that $+$ is
commutative and $\cdot$ distributes over $+$. We call these monoids
\emph{united} if $0 = 1$. To avoid confusion with semirings and lattices, we
will use $\varepsilon$ to denote the unit element of both monoids, that is,
$a + \varepsilon = a \cdot \varepsilon = a$. Note: we will often omit the
operator $\cdot$ and write simply $ab$ instead of $a \cdot b$, which is a usual
convention. We will further refer to $\varepsilon$ as \emph{empty}, the
operation~$+$ as \emph{overlay}, and the operation~$\cdot$ as \emph{connect}.

\newpage
\noindent
What can we tell about united monoids? First of all, it is easy to show that the
monoid $(S, +, \varepsilon)$ is idempotent:

\vspace{-5mm}
\begin{equation*}
\begin{array}{rcll}
a + a & = & a\varepsilon\ +\ a\varepsilon & \text{(unit of $\cdot$)}\\
 & = & a(\varepsilon\ +\ \varepsilon) & \text{(distributivity)}\\
 & = & a\varepsilon & \text{(unit of $+$)}\\
 & = & a & \text{(unit of $\cdot$)}
\end{array}
\end{equation*}
\vspace{-3mm}

\noindent
This means $(S, +, \varepsilon)$ is a commutative idempotent monoid, i.e.,
a \emph{bounded semilattice}.

The next property of united monoids is more unusual:
$ab = ab + a = ab + b = ab + a + b$. We call these equalities the \emph{containment laws}:
intuitively, when you connect $a$ and $b$, the constituent parts are contained
in the result $ab$. Let us prove containment:

\vspace{-5mm}
\begin{equation*}
\begin{array}{rcll}
ab\ +\ a & = & ab\ +\ a\varepsilon & \text{(unit of $\cdot$)}\\
 & = & a(b\ +\ \varepsilon) & \text{(distributivity)}\\
 & = & ab & \text{(unit of $+$)}\\
\end{array}
\end{equation*}
\vspace{-3mm}

\noindent
The two other laws are proved analogously (in fact, they are equivalent to each
other).

Surprisingly, the containment law $ab = ab + a$ is equivalent to $0 = 1$, i.e.,
the latter can be proved from the former:

\vspace{-5mm}
\begin{equation*}
\begin{array}{rcll}
0 & = & 1\cdot0 & \text{($1$ is the unit of $\cdot$)}\\
 & = & 1\cdot0\ +\ 1 & \text{(containment)}\\
 & = & 0\ +\ 1 & \text{($1$ is the unit of $\cdot$)}\\
 & = & 1 & \text{($0$ is the unit of $+$)}
\end{array}
\end{equation*}
\vspace{-3mm}

\noindent
This means that united monoids can be equivalently characterised by the
containment laws, which makes it possible to talk about
\emph{united semigroups}. The term ``united semigroup'' may sound somewhat
nonsensically (because semigroups have no ``units''), however, note that if they
secretly had unit elements, they would have to coincide.

In the same manner, the containment laws imply that zeroes of the two operations
must be the same. Let $z^{+}$ ans $z^{\cdot}$ denote the zeroes of $+$ and
$\cdot$, respectively. Then:

\vspace{-5mm}
\begin{equation*}
\begin{array}{rcll}
z^{+} & = & z^{\cdot}\ +\ z^{+} & \text{($z^{+}$ is a zero of $+$)}\\
 & = & z^{\cdot} \cdot z^{+}\ +\ z^{+} & \text{($z^{\cdot}$ is a zero of $\cdot$)}\\
 & = & z^{\cdot} \cdot z^{+} & \text{(containment)}\\
 & = & z^{\cdot} & \text{($z^{\cdot}$ is a zero of $\cdot$)}
\end{array}
\end{equation*}
\vspace{-3mm}

% \noindent
% Finally, let us prove another unusual property of united monoids: non-empty
% elements of $S$ can have no inverses. More precisely:

\subsection{Examples from this paper}

The containment laws should have reminded you about simplicial sets since
the latter are closed in terms of containment. For example, a filled-in triangle
contains its edges and vertices, and it cannot appear in a simplicial set
without any of them. This property can be expressed algebraically as:
$abc = abc + ab + ac + bc + a + b + c$. Interestingly, this ``3D'' containment
law follows from the ``2D'' version for united monoids:

\vspace{-5mm}
\begin{equation*}
\begin{array}{rcll}
abc & = & (ab + a + b)c & \text{(containment)}\\
 & = & (ab)c + ac + bc & \text{(distributivity)}\\
 & = & (abc + ab + c) + (ac + a) + (bc + b) & \text{(containment)}\\
 & = & abc + ab + ac + bc + a + b + c & \text{(commutativity)}
\end{array}
\end{equation*}
\vspace{-3mm}

\noindent
We can similarly prove higher-dimensional versions of the containment law; they
all follow from the basic law $ab = ab + a$, or, alternatively, from $0 = 1$.

Graphs, preorders and edge-labelled graphs also satisfy the containment laws.
In this paper we focused on non-empty structures, which is why we didn't come
across units of the overlay and connect operators. By wrapping trees in
\hcode{Maybe}, as we've done in a few cases, we were essentially augmenting our
structures with the unit \hcode{Nothing}.

It is worth remarking on the two remaining cases: sets and edge-labelled graphs.
When working with sets, we had only one operator \hdia, which was caused by the
fact that we used a trivial semiring with \zero~$=$~\one. One can think of this
case as \hdia~being a monoid united with itself.

The case of edge-labelled graphs is more interesting. There we had as many
operations as there were elements in the edge label semiring~\hcode{s}. It turns
out that every operation $\xrightarrow{x}$ where $x$ is non-zero is united with
\hdia. In other words, edge-labelled graphs are a
\emph{semiring of united monoids}.

\subsection{Other examples}

One example can be found in Haskell's \code{ApplicativeDo} language
extension~\cite{applicativedo}. It uses a simple cost model for defining the
execution time of programs composed in parallel or in sequence. The two monoids
are:

\begin{itemize}
    \item $(\mathbb{Z}^{\ge 0}, \max, 0)$: the execution time of programs $a$
    and $b$ composed in parallel is defined to be the maximum of their execution
    times:
    \[
    \textit{time}(a\ \text{\code{|}}\ b) = \max(\textit{time}(a),\ \textit{time}(b))
    \]

    \item $(\mathbb{Z}^{\ge 0}, +, 0)$: the execution time of programs $a$ and
    $b$ composed in sequence is defined to be the sum of their execution times:
    \[
    \textit{time}(a\ \text{\code{;}}\ b) = \textit{time}(a) + \textit{time}(b)
    \]
\end{itemize}

\noindent
Execution times are non-negative, hence both $\max$ and $+$ have unit $0$,
which is the execution time of the \emph{empty program}. It is easy to check
that distributivity ($+$ distributes over $\max$) and containment laws hold.
Note that the resulting algebraic structure is different from the tropical
max-plus semiring $(\mathbb{R}^{-\infty}, \max, -\infty, +, 0)$ commonly used in
scheduling, where the unit of $\max$ is $-\infty$ but the unit of $+$ is $0$.

Interestingly, parallel and sequential composition of programs also forms a
united monoid, where the empty program is the unit. One can therefore call the
cost model function $\textit{time}$ a \emph{united monoid homomorphism}.
A related example can be found in~\cite{beaumont2017concepts}, where
asynchronous circuit specifications are composed in parallel and sequentially,
using graphs labelled with Boolean predicates.

% What about tries?
% Equivalence classes?
% Languages

\newpage
\section{Related work}\label{sec-related-work}
This paper continues the research on \emph{algebraic graphs} started
in~\cite{mokhov_alga}. Since their introduction, algebraic graphs have been
implemented in several languages (including Agda, Fsharp, PureScript, Scala,
TypeScript, and R) and found application in industry, e.g. in GitHub's static
code analysis project Semantic~\cite{semantic_paper}\cite{semantic_repo}.
In this paper we further distill the essence of the algebraic approach to
working with graphs by simplifying the graph data type from four to just two
constructors, i.e., switching to using binary trees parameterised by a semiring.
One of the main limitations of the work on algebraic graphs was the lack of
edge labels and this paper successfully addresses it.

Other popular approaches for representing graphs in functional programming
languages can be found in~\cite{1995_king_graphs}\cite{1994_launchbury_st}
(adjacency arrays, available from the \textsf{containers} library)
and~\cite{2001_erwig_inductive} (so-called \emph{inductive graphs} where a graph
is decomposed into a \emph{context}, i.e., a node with its neighbourhood, and
the rest of the graph).

Using semirings as a general framework for solving problems on graphs is, of
course, not new. Dolan's ``Fun with Semirings''~\cite{2013_semirings_dolan} is a
notable example from the functional programming community, which motivated the
author to apply semirings in the context of algebraic graphs. A more recent
example~\cite{weighted_search} uses semirings for generalising algorithms for
weighted search.

% This paper does not discuss any algorithms

Jeremy Gibbons defines operations \code{above} and \code{beside} for composing
directed acyclic graphs~\cite{1995_gibbons_algebra}. They both have the same
unit (the empty graph) but the operation \code{beside} is defined only for
graphs of matching types. One can therefore consider this as an example of a
\emph{partial united monoid}. Another similar example is~\cite{typed_matrices},
where compatible matrices can be composed vertically or horizontally with the
empty matrix being the common unit.

Various flavours of parallel and sequential composition often form united
monoids. In a paper on Concurrent Kleene Algebra~\cite{hoare2011concurrent},
the authors use the term ``bimonoid'' to refer to such structures (without
investigating them in detail). In category theory, ``bimonoid'' is used for an
unrelated concept (a structure with a monoid and a
comonoid~\cite{porst_bimonoids}), hence our decision to use the term
``united monoids'', which appears to be unused and also more specific.

Rivas and Jaskelioff discuss various notions of computation as monoids in the
category of endofunctors~\cite{rivas_jaskelioff_2017}. Two of these monoids,
namely \hcode{Applicative} and \hcode{Monad}, turn out to have the same unit
\hcode{pure = return}, and they can therefore be considered \emph{united monoids
in the category of endofunctors}. Investigating categorical equivalents of
the presented ideas remains an interesting direction of future work.

% \acks
% I want to thank ....

\printbibliography
\end{document}
